\documentclass{article}
\usepackage{amsmath}
\usepackage{mathpartir}
\usepackage{ttquot}
\usepackage{pl-syntax}
\newcommand{\step}[0]{\ensuremath{\rightarrow}}
\newcommand{\bstep}[0]{\ensuremath{\Downarrow}}
\newcommand{\pstep}[0]{\ensuremath{\rightarrow_{\mathcal{S}}}}
\newcommand{\skips}[0]{~\textsf{skip}~}
\newcommand{\outputs}[0]{~\textsf{output}~}
\newcommand{\spawns}[0]{~\textsf{spawn}~}
\newcommand{\bop}{~\textsf{bop}~}
\newcommand{\uop}{~\textsf{uop}~}
\makeatletter
\newcommand{\conf}[1]{\langle #1\@confx}
\newcommand\@confx{\@ifnextchar\stopconf{\@confend}{\@confi}}
\newcommand\@confend[1]{\rangle}
\newcommand\@confi[1]{,#1 \@confx}
\makeatother

\begin{document}

\title{Concurrent Semantics for Pipeline DSL}
\author{Drew Zagieboylo}
\maketitle

It may be better to not define a new semantics - how about
we just use CHP (communicating sequential processes - https://avlsi.csl.yale.edu/act/doku.php?id=language:langs:chp).

The main features we care about in CHP are parallel composition ($\parallel$), sequential composition ($;$),
send ($C!V$), receive ($C?V$) and guarded execution ($@[G_1 \implies B_1 ... G_N \implies B_N]$)
where $G$ are guards and $B$ are programs.

\section{MetaVariables}
\begin{align*}
  \begin{array}{lc}
    \mbox{Integers} & n \\
    \mbox{Variables} & v,f\\
    \mbox{Expressions} & e\\
    \mbox{Statements} & c\\
    \mbox{Functions} & \mathcal{F}\\
    \mbox{Objects} & \mathcal{O}\\
  \end{array}
\end{align*}

Here is the intuition behind the syntax of this intermediate representation:\newline
Program:

List of Module Definition and Module Instantiations w/ Initial Conditions\newline
Module:

List of Stages, List of Locks, List of Formal Parameters (used to set initial conditions)\newline
Lock:

A special lock object with a few methods (acquire(), release(), reserve())\newline
Stage:

A set of receive statements, a sequential program generating combinational assignments,
and a set of guarded send statements\newline

TODOs:
\begin{itemize}
\item Make the data along channels be "Values" rather than just numbers, since 
we will actually be sending records along channels and reading out records into multiple variables.
\item Make "pred" (a.k.a. receives) conditional as well. This is tricky; since we haven't
received yet, we don't have any data that tells us whether or not to execute the receive.
Maybe we allow for conditional receives as commands? That makes things pretty ugly.
\item In general, I'm not super happy with how the communication semantics work right now;
need to clean them up.
\item Include Locks - they probably fit best as a different kind of channel;
so "succ" contains ``Lock.acquire()'' and "pred" contains ``Lock.release'' and
``Lock.reserve()''. See the point on conditional receives for handling the issue
of acquiring locks only if we need to.
\end{itemize}

\begin{figure}[h]
\begin{syntax}
  \abstractCategory[Variables]{v, x, z}
  \abstractCategory[Arrays]{v[], x[], z[]}
  \abstractCategory[Integers]{n, m}
  \abstractCategory[Locks]{L}
  \abstractCategory[Channels]{C}
  
  \category[Programs]{P}
  \alternative{ \{ List(M), List(v \mapsto n) \}} %Set of module declarations, Set of module instantiations, Set of Initial Conditions for each instantiated module

  \category[Modules]{M}
  \alternative{ \{List(S), List("Mem"), List(v)\}} %List of stages, first element is root stage; List of Locks; List of formal parameters set by initial conditions

  \category[Memories]{"mem"}
  \alternative{ \{L, v[]\}}

  \category[Stages]{S}
  \alternative{ \{List("pred"), "com", List("succ") \}}

  \category[Expressions]{e}
  \alternative{ e " binop " e }
  \alternativeLine{ " unop " e }
  \alternativeLine{ e " ? " e " : " e }
  \alternativeLine{ v }
  \alternative{ n }
  \alternative{ b }

  \category[Predecessors]{"pred"}
  \alternative{ v = "recv " C }

  \category[Commands]{"com"}
  \alternative{ "skip" }
  \alternative{ "com " ; " com" }
  \alternative{ v = e }
  \alternativeLine{ L".acquire()"}
  \alternative{ L".release()"}
  \alternative{ L".reserve()"}

  \category[Successors]{"succ"}
  \alternative{ "if ("v") send(" v, C")" }
  \alternative{ "if ("v") read mem("v")" }
  \alternative{ "if ("v") write mem("v, v")" }

\end{syntax}
\end{figure}
\section{Notes}

This language consists of a DAG of pipeline stages.
Every cycle, each stage will, in parallel, check their
predecessors for data. This is analagous to the idea
that registers and ready-valid transactions occur at the beginning
of clock boundaries, thus making the data available right away.
Then, all stages will, in parallel, execute combinational circuits
to compute output values and output guards. The values will
be sent along channels to successors and the guards will determine
which successors have data sent to them.
Then, each stage will, in parallel, send data along
the appropriate successor channels, iff it has received
data from all of its predecessors. If there is any predecessor
from which it hasn't received data, the sends will be skipped.
In the next cycle, this process will repeat except that stages
will remember which channels they have already received from and
will not receive from them again in the next cycle, until they
have sent data out on all expected channels.

Additionally, this language has a notion of locks or barriers,
which provide further requirements before a stage can send data.
For instance, some stage may execute acquire a lock
on a memory and then also send a request to read that memory.
Or it may simply acquire the lock without reading the memory.
They are separate notions in the semantics.

TODO:add these to the semantics and figure out if they should
be commands in the second phase of execution or extra guards in the send phase.

\begin{figure}
  \[\begin{array}{r|l} \hline
    \text{Variable Mappings} & \Sigma: S \rightarrow \sigma \\
    \text{State Variable Mappings} & \sigma: v \rightarrow n \\
    \text{Receive Status} & \Pi: S \rightarrow \pi \\
    \text{State Receive Status} & \pi: C \rightarrow b \\
    \text{Send Status} & \Delta: S \rightarrow \delta \\
    \text{State Send Status} & \delta: (C + "mem") \rightarrow b \\ \hline
  \end{array}\]
  \caption{Environment Types}
\end{figure}

\section{Operational Semantics}
Expression Semantics
\begin{mathpar}
  \inferrule*[Left={Uop}]{\conf{\sigma}{e}\stopconf \bstep n' \\ \uop n' = n''}
             {\conf{\sigma}{\uop n}\stopconf \bstep n''}\\
  \inferrule*[Left={Binop}]{\conf{\sigma}{e_1}\stopconf \bstep n \\ \conf{\sigma}{e_2}\stopconf \bstep n' \\ n \bop n' = n''}
             {\conf{\sigma}{e_1 \bop e_2}\stopconf \bstep n''}\\
  \inferrule*[Left={Int-Lookup}]{\sigma(v) = n}{\conf{\sigma}{v}\stopconf \bstep n}\\
  \inferrule*[Left={Ternary}]{\conf{\sigma}{e_b}\stopconf \bstep n_c \\ \conf{\sigma}{e_t}\stopconf \bstep n_t \\
  \conf{\sigma}{e_f}\stopconf \bstep n_f \\ n_c \neq 0 " ? " n_t " : " n_f}
             {\conf{\sigma}{e_b " ? " e_t " : " e_f}\stopconf \bstep n}\\
\end{mathpar}
Receive Semantics
\begin{mathpar}
  \inferrule*[Left={Block-Recv}]{ C.isEmpty() " or " \pi(C) = "true"}
             {\conf{\sigma}{\pi}{v = "recv " C}\stopconf \step \conf{\sigma}{\pi}\stopconf}\\
  \inferrule*[Left={Accept-Recv}]{!C.isEmpty() \\ \pi(C) = "false" \\ \sigma' = \sigma[v \mapsto C.peek()] \\
    \pi' = \pi[C \mapsto "true"]}
             {\conf{\sigma}{v = "recv " C}\stopconf \step \conf{\sigma'}{\pi'}\stopconf}\\
  \inferrule*[Left={Recv-List}]{
    \mathcal{P} = \{"pred"_0..."pred"_n\} \\
    \forall p \in \mathcal{P}. \conf{\sigma}{\pi}{p}\stopconf \step \conf{\sigma_p'}{\pi_p'}\stopconf \\
    \sigma' = \bigcup_{p \in \mathcal{P}} \sigma_p' \\ \pi' = \bigcup_{p \in \mathcal{P}} \pi_p'}
             {\conf{\sigma}{\pi}{\mathcal{P}}\stopconf \step \conf{\sigma'}{\pi'}\stopconf}\\
\end{mathpar}

Command Semantics
\begin{mathpar}
  \inferrule*[Left={Seq-Skip}]{ }
             {\conf{\sigma}{\skips ; "com"_2}\stopconf \step \conf{\sigma}{"com"_2}\stopconf}\\
  \inferrule*[Left={Seq}]{ \conf{\sigma}{"com"_1}\stopconf \step \conf{\sigma'}{"com"_1'}\stopconf}
             {\conf{\sigma}{"com"_1" ; com"_2}\stopconf \step \conf{\sigma'}{"com"_1'" ; com"_2}\stopconf}\\
  \inferrule*[Left={Assign}]{ v \not\in "dom"(\sigma) \\ \conf{\sigma}{e}\stopconf \bstep n \\ \sigma' = \sigma[v \mapsto n]}
             {\conf{\sigma}{v = e}\stopconf \step \conf{\sigma'}{\skips}\stopconf}\\
\end{mathpar}
Send Semantics
\begin{mathpar}
%  \boxed{\conf{\sigma}{\delta}{"if ("v_g") send(" v_v, C")"}\stopconf
%    \step \conf{\delta'}\stopconf}\\
  \inferrule*[Left={-Guard-False}]{  \sigma(v_g) = 0  \\ \delta' = \delta[C \mapsto "true"]}
             {\conf{\sigma}{\delta}{"if ("v_g")" ...}\stopconf \step \conf{\delta}\stopconf}\\
  \inferrule*[Left={Send-Blocked}]{ \sigma(v_g) \neq 0 \\ C.isFull() " or " \delta(C) = "true"}
             {\conf{\sigma}{\delta}{"if ("v_g" send(" v_v, C")"}\stopconf \step \conf{\delta}\stopconf}\\
  \inferrule*[Left={Send}]{ \sigma(v_g) \neq 0 \\ !C.isFull() \\ \delta(C) = "false"
    \\ \delta' = \delta[C \mapsto "true"] \\ C.push(\sigma(v_v))}
             {\conf{\sigma}{\delta}{"if ("v_g" send(" v_v, C")"}\stopconf \step \conf{\delta}\stopconf}\\
  \inferrule*[Left={Send-List}]
             { \forall c \in "dom"(\pi). \pi(c) = "true" \\
               \mathcal{S} = \{"succ"_0..."succ"_n\} \\
               \forall s \in \mathcal{S}. \conf{\sigma}{\delta}{s}\stopconf \step \conf{\delta_s'}\stopconf \\
             \delta' = \bigcup_{s \in \mathcal{S}} \delta_s' }
             {\conf{\sigma}{\pi}{\delta}{\mathcal{S}}\stopconf \step \conf{\delta'}\stopconf}\\
  \inferrule*[Left={Wait-List}]
             { \exists c \in "dom"(\pi). \pi(c) = "false" }
             {\conf{\sigma}{\pi}{\delta}{\mathcal{S}}\stopconf \step \conf{\delta}\stopconf}\\
\end{mathpar}

End of Cycle Semantics
\begin{mathpar}
  \inferrule*[Left={Retry}] {\exists c \in "dom"(\delta). \delta(c) = "false" }
             {\conf{\pi}{\delta}\stopconf \step \conf{\pi}{\delta}\stopconf} \\
  \inferrule*[Left={Reset}] {\delta = \{c_1 \mapsto "true"...c_n \mapsto "true"\} \\
    \pi = \{C_1 \mapsto "true"...C_n \mapsto "true"\} \\ \forall C_i. C_i.pop() \\
    \pi' = \{C_1 \mapsto "false"...C_n \mapsto "false" \} \\
    \delta' = \{c_1 \mapsto "false"..c_n \mapsto "false"\}}
             {\conf{\pi}{\delta}\stopconf \step \conf{\pi'}{\delta'}\stopconf}\\
\end{mathpar}
Module Semantics
\begin{mathpar}
  \inferrule*[Left={Execute-Recvs}]{
    \forall s \in S. \conf{\Sigma(s)}{\Pi(s)}{s."recv"}\stopconf \step
    \conf{\sigma_s'}{\pi_s'}\stopconf \\
    \Sigma' = \bigcup_{s\in S} \sigma_s' \\ \Pi' = \bigcup_{s \in S} \pi_s'}
             {\conf{\Sigma}{\Pi}{\Delta}{List(S)}\stopconf \step
               \conf{\Sigma'}{\Pi'}{\Delta}{\{List(S), "exec"\}}\stopconf} \\
  \inferrule*[Left={Execute-Commands}]{\forall s \in S. \conf{\Sigma(s)}{s."com"}\stopconf \step^*
    \conf{\sigma_s'}{\skips}\stopconf \\ \forall s \in S. \Sigma'[s \mapsto \sigma_s']}
             {\conf{\Sigma}{\Pi}{\Delta}{\{List(S), "exec"\}}\stopconf \step
              \conf{\Sigma'}{\Pi}{\Delta}{\{List(S), "send"\}}\stopconf}\\
  \inferrule*[Left={Execute-Sends}]{\forall s \in S. \conf{\Sigma(s)}{\Pi(s)}{\Delta(s)}{s."send"}\stopconf \step
    \conf{\delta_s'}\stopconf \\ \Delta' = \bigcup_{s \in S} \delta_s'}
             {\conf{\Sigma}{\Pi}{\Delta}{\{List(S),"send"\}}\stopconf \step
             \conf{\Sigma}{\Pi}{\Delta'}{\{List(S), "endcycle"\}}\stopconf}\\
  \inferrule*[Left={End-Cycles}]{ \forall s \in S. \conf{\Pi(s)}{\Delta(s)}\stopconf \step
  \conf{\pi_s'}{\delta_s'}\stopconf \\ \Pi' = \bigcup_{s \in S} \pi_s' \\ \Delta' = \bigcup_{s \in S} \delta_s' }
             {\conf{\Sigma}{\Pi}{\Delta}{\{List(S), "endcycle"\}}\stopconf \step
             \conf{\emptyset}{\Pi'}{\Delta'}{\{List(S), "recv"\}}\stopconf}\\
\end{mathpar}
\end{document}


In this semantics, each module is comprised of a set of stages.
In each cycle, every stage will, in parallel, check for inputs for validity
and outputs for readiness. If all of its inputs are ready, then it
will compute the values of its outpus and send any which are ready this cycle.
Every cycle, this procedure is repeated until all outputs have been sent.
On the cycle when a stage has sent all of its outputs, it will mark its
inputs as used (by popping them off of its input queues) and none
of its outputs as sent.
