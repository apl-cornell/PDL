\documentclass{article}

\usepackage{pl-syntax}


\title{Example Usage of \texttt{pl-syntax} Package}
\author{Andrew Hirsch \and Josh Acay}


\begin{document}
\maketitle

\section{The Simply Typed Lambda Calculus}

Read the comments in the \texttt{.tex} source for an explanation of how
the following is generated.

\begin{figure}[h]
  \begin{syntax}
    % Define a syntactic category with no right-hand side.
    \abstractCategory[Variables]{x, y, z}
    
    % Start a category definition. Optional input is the category
    % description, the other declares the meta variable(s) you will
    % use to refer to this category.
    \category[Expressions]{e}
    % Each alternative is introduces in a new line.
    \alternative{x}
    \alternative{\lambda x . e}
    \alternative{e_1 e_2}

    \category[Values]{v}
    \alternative{\lambda x . e}

    % You can add vertical spacing between categories to visually group them.
    \separate
    % You can pass the amount of space explicitly if you want to manually control it:
    % \separate[5ex]

    \category[Types]{\tau}
    \alternative{\mathbf{Nat}}
    % You can place alternatives on a new line.
    \alternativeLine{\tau_1 \rightarrow \tau_2}

    \category[Contexts]{\Gamma}
    \alternative{x_1 : \tau_1, \ldots, x_n : \tau_n}
    % You can use words to describe more complicated properties of
    % definitions if you don't want to use BNF.
    \note{no repeats}
    \note{unordered}
  \end{syntax}

  \caption{Terms and Types of the Simply Typed Lambda Calculus}
  \label{fig:abstract-syntax}
\end{figure}  

\end{document}
